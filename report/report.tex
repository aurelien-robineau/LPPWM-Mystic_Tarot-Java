\documentclass{article}
\usepackage[utf8]{inputenc}

\title{Project report - Mystic tarot}
\author{Aurelien Robineau}
\date{\today}

\begin{document}

\begin{titlepage}
\maketitle
\end{titlepage}

\section*{Preamble}
This is my report for the JAVA \textit{Mystic tarot} project. This project was
realized for the professional bachelor's degree \textit{Web and Mobile Project}
of \textit{Sorbonne University}.\\

\noindent
The intent of this report is to explain the architectural choices I made.

\section{Analysis and modelling}

\subsection{The \texttt{Card} class}
There are two types of card in the game of tarot: Minor Arcanas and
Major Arcanas. Most of the time, the mystic tarot only uses the Major
Arcanas, so I chose to implement only one \texttt{Card} class for the
Major Arcanas. It would be easy though to add Minor Arcanas just by
making the current class abstract and having a new class for each type
of card.\\

\noindent
Major Arcanas always have a number and a name. They must also have an
image to represent the physical card. So cards will have the following
properties:

\begin{itemize}
\itemsep0pt\parskip0pt\parsep0pt
\item a \texttt{number} that must be unique. It is the ID of the card
\item a \texttt{name}
\item a \texttt{imagePath} to the file of the card's image
\end{itemize}

\noindent
First I also put a \texttt{description} property but I realized later it was
unused so I removed it.

\subsection{Action classes}

I chose to add a class per action on cards so actions are all
independant and it is easy to add new ones.\\

\noindent
These classes - that I will call \textit{action classes} - will implement the
\texttt{CardAction} interface. This interface should have two methods:
one that actually do the action and another one that asks in the console
informations about the action to the user. However, I only added the
\texttt{openInConsole()} that opens a console interface because action classes
were too differents (actions with a different number of arguments, some have
sereval actions, etc\ldots{}). So I could not have a unique
\texttt{performAction(Card card)} method as I first intended.

\subsection{The \texttt{CardManager} class}

Any action on cards will go through another class, the
\texttt{CardManager}, wich calls itself methods from the action classes.
This way action classes are hidden and if a change has to be made, one
would only have to adapt the \texttt{CardManager} and not change
anything in the rest of the code. The class contains a method that
displays a menu in the console so the user can choose an action. It is
responsible for the cards so it contains the list of cards.\\

\noindent 
As there is only one card deck in the game because there is a unique
player, the \texttt{CardManager} is a singleton. This prevents to
instanciate several \texttt{CardManager} for nothing and makes it usable
globally via the \texttt{getInstance()} method, so we do not have to
pass it as an argument to each method. One could argue that the game may
have several players. In this case, we could remove the singleton
pattern and rename the \texttt{CardManager} class to \texttt{Deck} for
exemple. The \texttt{Deck} class could extend the \texttt{ArrayList}
class for more convenience. Then each player would have a \texttt{deck}
property and its own \texttt{Deck} instance. Since I only need one deck
I will use the \texttt{CardManager} solution.

\subsection{The \texttt{UserInput} class}

I added the \texttt{UserInput} class later in the project because I
realized I was doing the same verifications on the user input sereval
times. For exemple I must check that the user entered an integer or an
existing card number in different action classes.\\

\noindent
The \texttt{UserInput} class provides methods to ask the user to enter a
value while the value does not pass the method validation.\\

\noindent
For exemple the \texttt{getInteger()} method asks the user to enter a
value while it is not a integer.

\section{Basic card system}

It is requested to be able to add and delete cards, so I added to action
classes:

\begin{itemize}
\itemsep0pt\parskip0pt\parsep0pt
\item The \texttt{CardCreator} class
\item The \texttt{CardDeletor} class
\end{itemize}

\noindent
The \texttt{CardCreator} has a method that asks the user the values for
the new card and another one that creates the card from these values.\\

\noindent
The \texttt{CardDeletor} has a method that asks the user wich card to
delete and another one that actually delete the card.\\

\noindent
All action classes are protected so can only be used in the
\texttt{card} package, so the only entry point for other packages is the
\texttt{CardManager}.

\section{Extension and research}

\subsection{Editing cards}

To edit cards, I just need to add a new action class named
\texttt{CardEditor} wich has a method that asks the user wich card to
edit and then the new card values. It has another method that updates
the card values and saves it.

\subsection{Displaying cards}

To display the card list there is no need for an action class since the
\texttt{CardManager} constains the list of cards. I overwrote the
\texttt{toString()} method in the \texttt{Card} class and added a method
in the \texttt{CardManager} that simply loops through the cards and
displays them. However for more advanced display we could create a
\texttt{CardDisplayer} action class.

\subsection{Searching cards}

For the research feature I added another action class named
\texttt{CardSearcher}. It has a list of cards to search in and sereval
search methods:

\begin{itemize}
\itemsep0pt\parskip0pt\parsep0pt
\item one for searching a card by its \texttt{number}
\item one for searching a card by its \texttt{name}
\item one for searching cards by there \texttt{number} and \texttt{name} in the
same time
\end{itemize}

\subsection{Adding a image}

For the image I already had the \texttt{imagePath} property on cards so
I just added a new field to be asked to the user in the
\texttt{CardCreator} and \texttt{CardEditor} classes. The user must give
the path to the image. I do not yet have an \texttt{image} property
containing the image file because the image will not be rendered for
now.

\section{Saving cards}

\subsection{Classic serialization}

I added a new class named \texttt{CardSerializer} to serialize and save
cards to files. The constructor takes a \texttt{Card} as an argument
that will be the card to serialize. The \texttt{CardSerializer} has two
methods to manage this card. One to create the card file, the other one
to delete it. It also has a static method to load all cards from serial
files.\\

\noindent
Files are saved in the \texttt{data/cards/serialized} directory. They are named
\texttt{<card\_number>.serial}. Each file in this directory will be loaded when
starting the application. Invalid or old files found when leaving the app will
be deleted.

\subsection{JSON serialization}

To save cards as JSON I added the exact same methods but with the JSON
logic using the Gson API.\\

\noindent
JSON files are saved in the same directory than the binary files. They are named
\texttt{<card\_number>.json}.

\noindent
Both binary and JSON serialization are still available. It is possible
to switch method by using the \texttt{SERIALIZATION\_METHOD} property of
the \texttt{CardSerializer} class.

\section{Graphical User Interface}

\subsection{\texttt{gui} package}

The card management files are all in the \texttt{card} package. To
seperate clearly the GUI from the card management I added a new package
\texttt{gui}. Since each action class is \texttt{protected}, the GUI can only
rely on the \texttt{CardManager} singleton to interact with cards.
Indeed the GUI does not have to know the logic behing the scenes, but
must only call simple methods to manage cards.

\subsection{The \texttt{GuiManager} class}

For the same reasons I used a \texttt{CardManager} class, I created a
\texttt{GuiMananger} class. Any action on the GUI will use this class
wich contains the list of cards to display. So no action will be done
directly into the action listeners. Again, this separation makes
maintenance more easy and the application more scalable.\\

\noindent
Since there is only one list of cards to display, the
\texttt{GuiManager} is also a singleton and can easily be used from
anywhere in the GUI.

\subsection{Package architecture}

We will need frames to interact with the user. Theses frames will
contain components (buttons, inputs, file inputs, etc\ldots{}) wich will
trigger actions via action listeners.\\So the GUI package will have three
sub-packages:

\begin{itemize}
\itemsep0pt\parskip0pt\parsep0pt
\item \texttt{gui/frames}
\item \texttt{gui/components}
\item \texttt{gui/listeners}
\end{itemize}

\newpage
\subsection{The main frame}

I decided that the main frame will have:

\begin{itemize}
\itemsep0pt\parskip0pt\parsep0pt
\item an area to display the cards
\item a button to create a new card
\item a text input to search cards
\end{itemize}

\noindent
The input will update the shown cards everytime the user types in it and
will show all cards if empty.\\

\noindent
Each card will be displayed in its own panel with buttons to delete and
edit the card.\\

\noindent
The create and edit card forms will be displayed in there own frames.

\subsection{Card images}
When adding a image to a card, the image is copied to the
\texttt{data/cards/images} directory. The card \texttt{imagePath} is set to the
relative path of the new image so cards can be shared on different devices.
When editing or deleting a card, the old image is removed from the directory.

\subsection{Switching to console interface}
The console interface is still available. It is easy to switch from a interface
to the other with the \texttt{INTERFACE} property of the \texttt{MysticTarot}
class.

\end{document}